\documentclass{article}
\begin{document}
\title{KE2D}
\author{J. R. C. King}
\maketitle
\begin{abstract}
This document contains a hastily written outline of how to run KE2D
\end{abstract}


\section{File structure}
KE2D.tar contains four folders: SOURCE, RUN, PLOT and DOCUMENTATION, and three files: EDITCODE, COMPILECODE, RUNCODE.

\subsection{SOURCE}
The folder SOURCE contains the following files
\begin{itemize}
\item \textbf{commonblock} - Declarations of all common variables.
\item \textbf{EVOLVE.f} - Single phase Euler solving subroutine.
\item \textbf{SWEEP1D.f} - A component of the single phase Euler solver.
\item \textbf{RIEMANNHLLC.f} - Solves a single phase Riemann problem to provide HLLC fluxes.
\item \textbf{WENOZ.f} - Weighted essentially non-oscillatory reconstruction.
\item \textbf{KINGEULER2D.f} - The main program.
\item \textbf{SETTSTEP.f} - Sets the time step based on the CFL condition.
\item \textbf{GFMSPLIT.f} - Creates two single-fluid domains for the GFM.
\item \textbf{FINDPAIR.f} - Finds neighbours for interface cells.
\item \textbf{FINISH.f} - An unnecessary subroute to end the program.
\item \textbf{DUMPRESULTS.f} - Outputs data when required.
\item \textbf{LEVELSET.f} - Updates the level set.       
\item \textbf{SETUP.f} - Sets up initial conditions.
\item \textbf{NLAABC.f} - Applies the NLAA boundary condition.
\item \textbf{NLAABCGHOST.f} - Applies the NLAA boundary condition with the ghost.
\item \textbf{SETBCELL.f} - Determines which cells are interface cells.
\item \textbf{VORTICITY.f} - Calculates the vorticity in a quite simplistic manner.
\end{itemize}

\subsection{RUN}
The folder RUN contains the following
\begin{itemize}
\item \textbf{KE2D} - the executable.
\item \textbf{init.params} - the main parameter file/
\item \textbf{AG.dat} - initial conditions for an air gun bubble.
\end{itemize}
When KE2D is run, all output files (with the extension .out) will be created in the folder RUN.

\subsection{PLOT}
The folder PLOT contains a sample Octave scripts to load the data.
\begin{itemize}
\item \textbf{indata.m} - this will load all the data, provided 'nx1', 'nx2' and 'outf' are set to match the values in init.params.
\end{itemize}

\subsection{DOCUMENTATION}
The folder documentation contains this document and the Latex source files.

\section{Output format}

All output files are written in ASCII in double precision, which is probably quite extravagant.
\begin{itemize}
\item \textbf{dt.out} - time, time step value
\item \textbf{pbound.out} - time, pressure at domain boundary
\item \textbf{pint.out} - time, interface pressure
\item \textbf{interface.out} - time, interface position
\item \textbf{rhobound.out} - time, density at domain boundary
\item \textbf{ubound.out} - time, velocity at domain boundary
\item \textbf{bmass.out} - time, bubble mass
\item \textbf{bz.out} - time, z-coordinate of bubble centre of mass
\item \textbf{A.out} - level set
\item \textbf{E.out} - energy
\item \textbf{P.out} - pressure
\item \textbf{rho.out} - density
\item \textbf{ux.out} - x-component of velocity
\item \textbf{uz.out} - z-component of velocity
\item \textbf{ux1.out} - r-component of velocity
\item \textbf{ux2.out} - $\theta$-component of velocity
\item \textbf{vort.out} - vorticity
\item \textbf{x.out} - x coordinate
\item \textbf{z.out} - z coordinate
\item \textbf{x1.out} - r coordinate
\item \textbf{x2.out} - $\theta$ coordinate
\end{itemize}

The files with 2 columns and time in the first column can just be plotted with gnuplot. pint.out and interface.out for example.

The files for arrays of properties (eg. P.out) are written by looping over the $\theta$-index within a loop over the r-index. This is repeated every time the output is written. If the spatial index in the r-direction is $i$ and in the $\theta$-direction is $j$, and the time index is $k$, then the first row of P.out is for $\left(i,j,k\right)=\left(1,1,1\right)$, the second is $\left(i,j,k\right)=\left(1,2,1\right)$. After one full sweep through $j$, it starts at $\left(i,j,k\right)=\left(2,1,1\right)$, then $\left(i,j,k\right)=\left(2,2,1\right)$, etc. After the full sweep through $i$, it is $\left(i,j,k\right)=\left(1,1,2\right)$, then $\left(i,j,k\right)=\left(1,2,2\right)$, etc. The file /PLOT/indata.m loads all these outputs into 3D arrays in Octave.

The final 4 files listed contain the coordinates - there are two choices; Cartesian or polar. It tends to be easier to plot things against Cartesian coordinates. These files are written by looping over the $\theta$-index within a loop over the r-index.

The code will also create a series of files 'DUMP00001.DUMP', 'DUMP00XXX.DUMP' etc, every XXX-1 time-steps. The first line of these contains the time. The subsequent lines contain all the arrays required to restart the simulation (except the mesh...) - density, pressure, radial velocity, polar velocity, total energy, level-set. These are written by looping over the $\theta$-index within a loop over the r-index.
\section{Parameters}

Parameters: here's what you can change in init.params.
\begin{itemize}
\item \textbf{nx1} - number of cells in radial direction
\item \textbf{nx2} - number of cells in $\theta$ direction
\item \textbf{nt} - number of time steps
\item \textbf{outfreq} - output fields every outfreq timesteps
\item \textbf{x1d} - domain radius
\item \textbf{x2d} - angle of domain
\item \textbf{CFL} - courant condition
\item \textbf{coordsno} - the coordinate system - $0=$Cartesian, $1=$cylindrical, $2=$spherical.
\item \textbf{grav} - the value of gravitational acceleration
\item \textbf{viscflag} - a flag for viscosity. $0=$inviscid, $1=$viscous
\item \textbf{ghostflag} - a flag for the ghost $0=$no ghost, $1=$ghost
\item \textbf{d} - the depth of the bubble in metres
\item \textbf{dumpflag} - a flag to decide whether to start from scratch. 0=from scratch, 1=from dump
\item \textbf{boundflag} - a flag to set the outer bc. $0=$wall, $1=$NLAA
\item \textbf{distflag} - a flag to set forcing on initial shape. $0=$no forcing, $1=$sin, $2=$cos.
\end{itemize}

AG.dat file just contains a list of left and right properties plus the initial interface location.

\section{How to run}
In the folder KE2D there are three files:

\begin{itemize}
\item \textbf{./EDITCODE} will open all the source files, paramater files and Octave scripts in gedit.
\item \textbf{./COMPILECODE} will delete output files and executable from RUN, compile the code with gfortran, and put the new executable 'KE2D' into the folder RUN
\item \textbf{./RUNCODE} will run the code!
\end{itemize}

As you receive the code, it will run an air gun bubble simulation on a 1metre domain with 50x50 cells for 3e4 time steps - about 1 oscillation. These initial conditions produce the 2D trace in Figure 6.1 of Chapter 6 in my thesis.

The simulation is very roughly (with some artistic license in applying a reduction factor to the initial pressure) equivalent to a 250cu.in. air gun at 2000psi in 7.7m of water.

\end{document}



